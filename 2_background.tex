%TODO: include the ref to bbob functions
\section{Tournament Selection}\label{sec:background:Selection Scheme} 

The selection scheme is intended to improve the average quality of the GA by giving solutions of higher quality a higher probability of composing the next generation~\cite{blickle1995mathematical}.

Goldberg and Deb~\cite{goldberg1991comparative} stated that there exists several selection schemes commonly used in GA, such as the rank selection, the roulette-wheel selection and the tournament selection. They also declare that the complexity of the tournament selection is O(n) and say that tournaments are often held with tournament size s = 2, although larger tournament sizes may be considered.

\subsection{The Tournament Selection Procedure}\label{sec:background:tournament_selection} 

Tournament selection is translation and scaling invariant~\cite{maza1993analysis}. That means that the behavior of the selection method is not affect by the scale of the fitness values~\cite{back2000evolutionary}.

Tournament selection is conducted as: choose a number $K$ of solutions randomly from the population and select the best solution from this group as a parent for the crossover operator. Repeat this operation for each parent necessary. This $K$ is the tournament size parameter that we investigate in this paper.
%
%
%\begin{algorithm}
%	\caption{Tournament Procedure}
%	\label{tournament_procedure}
%	\SetKwInOut{Input}{input}\SetKwInOut{Output}{output}
%	\SetKw{return}{return}
%	\Input{individuals, k, tournsize}
%	\tiny	\begin{itemize}
%		\item individuals: A list of individuals to select from.\vspace{-3mm}
%		\item k: The number of individuals to select.\vspace{-3mm}
%		\item tournsize: The number of individuals participating in each tournament.\vspace{-3mm}
%	\end{itemize}
%	
%	\normalsize
%	
%	\Output{list of selected individuals}
%	\BlankLine
%		
%	selected = list()\\
%	\For{i = 1 to k}{
%		candidates = choose\_Random(individuals, tournsize)\\
%		selected[i] = max(candidates, by: fitness value)
%	}
%	\return selected
%\end{algorithm}
%
%
%
%Often tournaments are held only between two individuals (binary tournament) but a generalization is possible to an arbitrary group size k called tournament size~\cite{blickle1995mathematical, goldberg1991comparative}. A tournament selection, as any selection mechanism in GA, simply favors the selection of better individuals of a population to influence the next generation. 


\subsection{Selection Schemes and BBOB functions}


Holtschulte and Moses, in~\cite{holtschulte2013benchmarking} used rank selection in their GA implementation and applied it to the BBOB functions. They declared that rank selection should be replaced by an elitism selection scheme to improve the performance of their GA implementation by spreading of high quality solutions through the population. 

One of these elitist selection scheme is the fitness-based roulette-wheel. However, it has been shown to lead to premature convergence~\cite{baker1987reducing}. Nicolau, ~\cite{nicolau2009application} uses the tournament selection in his GA aiming to solve the BBOB functions. He declares ``for harder problem domains, such as the ones on the BBOB-Benchmark suite, the tournament selection scheme is more appropriate.''.

Then, in this work we chose to analyze the tournament selection behavior in the BBOB functions.


Also, tournament selection is said to be simple to code, efficient in both parallel and non parallel architectures, and its pressure is able to be adjusted for different domains~\cite{miller1995genetic}.



\subsection{Tournament Pressure}\label{sec:background:tournament_pressure} 
Muhlenbein et al., in~\cite{muhlenbein1993predictive}, introduced the concept of selection intensity, which is to measure the pressure of selection schemes. The selection intensity is defined as the expected average fitness of the population after selection.  

Miller et al.~\cite{miller1995genetic}, argued that with higher tournament sizes, the selection pressure rises resulting in higher convergence rates, because the convergence rate of a GA is largely determined by the selection pressure.  Accordingly, if the selection pressure is too low, the convergence rate will be slower if it is too high, there is an increased chance of the GA prematurely converging. 

We understand that it is important to explore the values for tournament size, since this value seems to determine the balance of exploration and exploitation, by changing the selection intensity. It is possible that this balance could change given a domain, once it is expected that every domain holds a specific characteristic that would need a different proportion of this balance.

\subsection{Tournament Size in the Literature}\label{sec:background:tournament_size} 

In 1991, Goldberg~\cite{goldberg1991real}, analyzed the tournament selection with size 2 in a real-coded GA. He recognized, empirically, that after some time only individuals with relatively high function values will be represented in the population. Also he compares the importance of mutation operators and the tournament selection operator and states that ``the addition of some small amount of creeping mutation or other genetic operators should not materially affect these results''.


Agrawal~\cite{agrawal1995simulated}, in 1995, implemented real-coded GA with tournament selection with size 2. Their goal was to compare the results of the simulated binary crossover (SBX) with the binary single-point crossover and the blend crossover. They concluded that the SBX has search power similar to the single-point crossover and the SBX performs better in difficult test functions than the blend crossover. 

In 1999, Tsutsui et al.~\cite{tsutsui1999multi} proposed a simplex crossover (SPX), a multi-parent recombination operator for real-coded genetic algorithms. For their experiments, they used a real-coded GA with tournament selection with size 2 and showed that ``SPX works well on functions having multi modality and/or epistasis with a medium number of parents: 3-parent on a low dimensional function or 4 parents on high dimensional functions.''. 

Later, in 2000, Deb~\cite{deb2000efficient}, implemented a binary GA and a real-coded GA, with tournament size 2. He aimed to devise a penalty function approach that does require any penalty parameter. Because all of their search spaces are defined in the real space, the real-coded GA was more suited in finding feasible solutions, but in all cases, this approach `` [has] a niche over classical methods to handle constraints''.


In 2001, Beyer et al.~\cite{beyer2001self}, analyzed the results of using the simulated binary crossover (SBX), the fuzzy recombination operator and the blend crossover (BLX), all with no mutation and tournament size 2 in a real-coded GA. The results indicated that those crossovers exhibit similar performance and linear convergence order.


Bhunia et al.~\cite{bhunia2009application}, in 2009, created a tournament GA application to solve an economic production lot-size (EPL) model. 

Sawyerr et al.~\cite{sawyerr2015benchmarking}, in 2015, proposed the RCGAu, hybrid real-coded genetic algorithm with ``uniform random direction'' search mechanism. They used tournament selection with size 3 to explore the suite of noiseless black-box optimization testbed and the results showed great performance in some functions as $f_7$ and $f_{21}$.

Goldberg et AL~\cite{goldberg1993toward}, in 1993, implemented a binary GA using the tournament selection as a convenient way to control selection pressure. They wanted to study the theory aspects of the GA focusing in the mixing building block. The actual effect of the tournament value was not investigated empirically in the work.

In 1999, Harik et al.~\cite{harik1999compact} studied a simple GA and experimented with the tournament size. Their work shows that a given solution quality can be obtained faster by the tournament size equal to 4 or 8, instead of tournament size equal to 2 or 40. 

Also, Harik et al.~\cite{harik1999gambler} implement a binary coded GA using tournament selection. They analyze the impact of the tournament size on their GA with a 100-bit one-max function. They inspected this implementation with tournament sizes of 2, 4, and 8. They observed that the proportion of the building blocks changed with the tournament size, being higher for smaller tournament sizes.

In 2007, Kaelo et al.~\cite{kaelo2007integrated} compared the usage of the tournament size in a real-coded GA. They discuss the impact of using tournament size of 2 and tournament size of 3. The results of comparing these two tournament sizes indicated that the tournament with size 3 is more greedy, reduces the number of functions evaluations greatly and also improves CPU usage, therefore, it impacts negatively in the success rate of their GA.


Also in 2009, Nicolau~\cite{nicolau2009application} implemented a binary GA to solve the noise-free BBOB 2009 testbed. Trough a quick experimentation with a subset of benchmark functions which yielded the following equation:

\vspace{-7.5mm}
\begin{equation}
k = P/500, \text{ where P is the population size.}
\end{equation}


From the works showed here, it is possible to understand that there is a preference to small values for the tournament size. We traced back to 1991, and we found out that it has been a popular choice ever since. We must highlight that these values are chosen usually chosen without much justification.




