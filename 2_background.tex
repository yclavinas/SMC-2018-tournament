%TODO: include the ref to bbob functions
\section{The Tournament Selection}\label{sec:background:Selection Scheme} 



A selection scheme is a process that aims to improve the average quality of the GA by giving solutions of higher quality a higher probability of composing the next generation~\cite{blickle1995mathematical}.  The tournament selection, as any selection mechanism in GA, simply favors the selection of better individuals of a population to influence the next generation.

Tournament Selection is a selection scheme that has a unique set of characteristics that makes it a very popular choice~\cite{blickle1995mathematical}. It is simple to code, and its complexity is O(n) on the size of the population~\cite{goldberg1991comparative} which makes it efficient for both nonparallel and parallel architectures.

 The tournament selection is also translation and scaling invariant~\cite{maza1993analysis}, which means that the behavior of the selection scheme is not affected by the scale of the fitness values~\cite{back2000evolutionary}.


It is often claimed that the tournament selection is able to adjust the selection pressure by changing its only parameter, which controls the number of individuals chosen to take part in the tournament. That is an important feature, because any selection scheme should have the capacity to adjust its selection pressure given different domains. When only two individuals are selected, it is often called a binary tournament~\cite{blickle1995mathematical, goldberg1991comparative}, which is a popular choice. However, in principle there is no limit for the size of the tournament.

\subsection{The Tournament Selection Procedure}\label{sec:background:tournament_selection} 



Tournament selection is conducted as: choose a number $K$ of solutions randomly from the population and select the best solution from this group as a parent for the crossover operator. Repeat this operation for each parent necessary. This $K$ is the tournament size parameter that we investigate in this paper.




\subsection{Selection Schemes and BBOB functions}



Holtschulte and Moses~\cite{holtschulte2013benchmarking} used rank selection in their GA implementation and applied it to the BBOB functions. They declared that rank selection should be replaced by an elitism selection scheme to improve the performance of their GA implementation by spreading of high quality solutions through the population. 

The roulette-wheel and the tournament selection are two of these elitist selection schemes. However, it has been shown that roulette-wheel leads to premature convergence~\cite{baker1987reducing}. 

Nicolau, ~\cite{nicolau2009application} used the tournament selection in his GA with the goal of solving the BBOB functions. He declares ``for harder problem domains, such as the ones on the BBOB-Benchmark suite, the tournament selection scheme is more appropriate.''.

These two studies in addition to the tournament selection popularity and its power to be adjusted for different domains, give us motive to analyze the tournament selection behavior considering the BBOB functions.


\subsection{Tournament Pressure}\label{sec:background:tournament_pressure} 
Muhlenbein et al., in~\cite{muhlenbein1993predictive}, introduced the concept of selection intensity, which is to measure the pressure of selection schemes. The selection intensity is defined as the expected average fitness of the population after selection. This definition can only be applied because the tournament selection is scale and translation invariant~\cite{blickle1995mathematical}.

Blickle et. al~\cite{blickle1995mathematical}, declare that the selection intensity can be analytically calculate for tournament sizes of 2 and 3 and for larger values with numerical integration. They calculate the selection intensity for values between 2 and 30 and showed that with higher values for the tournament size the selection intensity increases. This calculation allows the prediction of the convergence time and to can be used to compare different selection schemes.

Blickle et .al~\cite{blickle1995comparison} compared some selection schemes, given their selection intensity. They concluded that for the same selection intensity the tournament selection has the smallest loss of diversity and the highest selection variance when compared with the truncation selection, the linear ranking, the exponential rank, and the proportional rank.

Miller et al.~\cite{miller1995genetic}, argued that with higher tournament sizes, the selection pressure rises resulting in higher convergence rates, because the convergence rate of a GA is largely determined by the selection pressure.  Accordingly, if the selection pressure is too low, the convergence rate will be slower if it is too high, there is an increased chance of the GA prematurely converging.

Based on the studies from above, we understand that the convergence directly influences the the final results of a GA, once any variation on the convergence rates changes the lost of diversity, and the selection variance. Consequently, it is important to explore the values for tournament size, since this value seems to determine the balance of exploration and exploitation, by changing the selection intensity. It is possible that this balance could vary given a domain, once it is expected that every domain holds a specific characteristic that would need a different proportion of this balance.

\subsection{Tournament Size in the Literature}\label{sec:background:tournament_size} 

In 1991, Goldberg~\cite{goldberg1991real}, analyzed the tournament selection with size 2 in a real-coded GA. He recognized empirically that after some time only individuals with relatively high function values would be represented in the population. He recognizes that selection may emphasize highly fit individuals leading to a small number of alternatives considered.

Agrawal~\cite{agrawal1995simulated}, in 1995, implemented real-coded GA with tournament selection with size 2. Their goal was to compare the results of the simulated binary crossover (SBX) with the binary single-point crossover and the blend crossover. They concluded that the SBX has search power similar to the single-point crossover and the SBX performs better in difficult test functions than the blend crossover. They understand that for some functions the selection pressure is not enough.

In 1999, Tsutsui et al.~\cite{tsutsui1999multi} proposed a simplex crossover (SPX), a multi-parent recombination operator for real-coded genetic algorithms. For their experiments, they used a real-coded GA with tournament selection with size 2 and showed that ``SPX works well on functions having multi modality and/or epistasis with a medium number of parents: 3-parent on a low dimensional function or 4 parents on high dimensional functions.''. 

Later, in 2000, Deb~\cite{deb2000efficient}, implemented a binary GA and a real-coded GA, with tournament size 2. He aimed to devise a penalty function approach that does not require any penalty parameter. Because all of their search spaces are defined in the real space, the real-coded GA was more suited in finding feasible solutions, but in all cases, this approach `` [has] a niche over classical methods to handle constraints''. The concluded with declaring that some parametric study should be done on the important parameters to improve the performance.

%Bhunia et al.~\cite{bhunia2009application}, in 2009, created a tournament GA application to solve an economic production lot-size (EPL) model. In their study, the tournament selection is used to produce a better population from the existing two populations. 


In 2001, Beyer et al.~\cite{beyer2001self}, analyzed the results of using the simulated binary crossover (SBX), the fuzzy recombination operator and the blend crossover (BLX), all with no mutation and tournament size 2 in a real-coded GA. The results indicated that those crossovers exhibit similar performance and linear convergence order.


Sawyerr et al.~\cite{sawyerr2015benchmarking}, in 2015, proposed the RCGAu, hybrid real-coded genetic algorithm with ``uniform random direction'' search mechanism. They used tournament selection with size 3 to explore the suite of noiseless black-box optimization testbed and the results showed great performance in some functions as $f_7$ and $f_{21}$.

Goldberg et AL~\cite{goldberg1993toward}, in 1993, implemented a binary GA using the tournament selection as a convenient way to control selection pressure. They wanted to study the theory aspects of the GA focusing in the mixing building block. The actual effect of the tournament value was not investigated empirically in the work.

In 1999, Harik et al.~\cite{harik1999compact} studied a simple GA and experimented with the tournament size. Their work shows that a given solution quality can be obtained faster by the tournament size equal to 4 or 8, instead of tournament size equal to 2 or 40. 

Also, Harik et al.~\cite{harik1999gambler} implement a binary coded GA using tournament selection. They analyze the impact of the tournament size on their GA with a 100-bit one-max function. They inspected this implementation with tournament sizes of 2, 4, and 8. They observed that the proportion of the building blocks changed with the tournament size, being higher for smaller tournament sizes.

In 2007, Kaelo et al.~\cite{kaelo2007integrated} compared the usage of the tournament size in a real-coded GA. They discuss the impact of using tournament size of 2 and tournament size of 3. The results of comparing these two tournament sizes indicated that the tournament with size 3 is more greedy, reduces the number of functions evaluations greatly and also improves CPU usage, therefore, it impacts negatively in the success rate of their GA.


Also in 2009, Nicolau~\cite{nicolau2009application} implemented a binary GA to solve the noise-free BBOB 2009 testbed. Trough a quick experimentation with a subset of benchmark functions which yielded the following equation:

%\vspace{-7.5mm}
\begin{equation}
k = P/500, \text{ where P is the population size.}
\end{equation}


From the works reviewed here, it is possible to understand that there is a preference to small values for the tournament size. We traced back to 1991, and we found out that it has been a popular choice ever since. We must highlight that these values are usually chosen without much justification.




