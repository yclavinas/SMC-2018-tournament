%comentario para claus re-escrever
%TODO
\section{Conclusion}
\label{sec:conclusion}


We proposed an experimental analysis of the impact of the tournament
size on the BBOB benchmark functions. We verified that there are few
mathematical or experimental studies that supports a choice of value
for the tournament size. Usually, the tournament size is chosen to be
2 or 3, based on ``previous results''.
 
We analyzed a group of tournament size values on different selection
operators. From the results of this experiment, we found that using
the tournament size with values like 2 or 3 may not always be a good
choice, depending on the characteristic of the problem being explored.

Our experiments demonstrated that for each problem domain the effect
of tournament size on the final result can have completely different
characteristics. In some cases a higher tournament size achieves the
best result, in others, the opposite happens. We also observed
that even for the same problem, the choice of tournament size may
depend on the dimensionality or the structure of the Genetic Algorithm
being used.

In conclusion, we suggest that future works consider the use of
self-adaptive schemes for the control of the tournament size
parameter, instead of relying on fixed values. We believe that using
this technique may diminish the time spend on the design process, and
minimize the impact of bad choice of values for the tournament size by
adjusting this parameter on-line during the search process.

Some of the limitations of the current experiment are the limited
range of tournament size values tested, and the analysis of iteraction
effects between the tournament size and other parameters of the GA,
including the choice of operators.  Overcoming these limitations would
allow the community to better understand the role of the tournament
size in the quality of the search.



