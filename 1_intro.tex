%%%%%%%%%%%%%%%%%%%%%%%%%%%%%%%%%%%%%%%%%%%%%%%%%%%%%%%%%%%%%%%%%%
\section{Introduction}\label{intro}

 Genetic Algorithm (GA) is a well-known meta-heuristic that is used to seek solutions given a search space. The GA relies on operators such as the crossover operator, the mutation operator and the selection operator. These operators have their own and specific role in the GA. The selection operator selects solutions to compose the next generation given their measured fitness function quality value, with higher solutions having higher probability to compose the next generation. The crossover operator generates new solutions by recombining pairs of good solutions to generate new ones. The mutation operator is applied to the population to sustain diversity, by changing the value of a part of a solution. Both crossover and mutation operators are applied to explore new points in the search space~\cite{blickle1995mathematical}.

One of the most used selection operator is the tournament selection operator. It selects solutions based on their fitness value and on one parameter, the tournament size. Controlling the tournament size parameter is argued to be one way of adjusting the balance between the exploration and explotation~\cite{blickle1995mathematical}, which governs the search process in GA. Although it is a very important balance, it cannot be set directly~\cite{filipovic2012fine}. The selection pressure of tournament selection is expected to increase as the tournament size becomes larger. That is, the higher the selection pressure, the more the better solutions influences the next generation~\cite{miller1995genetic}. 

As a rule of thumb, small values, such as 2 or 3, are widely used. Filipovic~\cite{filipovic2012fine} argued that small vales lead to small selection pressure and may be good choices, but, very often, the search process converges too slowly with smaller tournament size (and too fast with bigger tournament size). 

 We would like to reconsider the idea that exists a relationship between exploration versus exploitation and the tournament size parameter by verifying, experimentally, the real impact of the tournament size on the Genetic Algorithm.
 
 Consequently, our goal in this paper is to observe the performance of the tournament selection in a real-valued Genetic Algorithm. For that we applied the Genetic Algorithm to 24 noise-free BBOB benchmark functions with 10, 20 and 40 dimensions~\cite{hansen2010real}. We explore the number of solutions to be selected by the Tournament operator, with values from 2 to 25. Then we analyze the results to verify any relationship among the tournament size value, the benchmark functions, and the performance.

%TODO: Add here the referes for "from the literature
 A review of the literature shows that there is a preference to small values for the tournament size, as 2 or 3 ~\cite{goldberg1991real, goldberg1993toward, agrawal1995simulated, harik1999gambler, tsutsui1999multi, harik1999compact, deb2000efficient, beyer2001self,kaelo2007integrated, bhunia2009application,  nicolau2009application, sawyerr2011comparative, sawyerr2015benchmarking}. This has been a popular choice since 1991. Most of the papers use little scientific justification for their choice for the tournament size value. On the other hand, Nicolau~\cite{nicolau2009application} tried to define empirically the value of the tournament size for his binary GA.
 
 