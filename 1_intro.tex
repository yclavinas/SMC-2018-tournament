%%%%%%%%%%%%%%%%%%%%%%%%%%%%%%%%%%%%%%%%%%%%%%%%%%%%%%%%%%%%%%%%%%
\section{Introduction}\label{intro}

% Genetic Algorithm (GA) is a well-known meta-heuristic that is used to
% seek solutions given a search space. The GA relies on operators such
% as the crossover operator, the mutation operator and the selection
% operator. These operators have their own and specific role in the
% GA. The selection operator selects solutions to compose the next
% generation given their measured fitness function quality value, with
% higher solutions having higher probability to compose the next
% generation. The crossover operator generates new solutions by
% recombining pairs of good solutions to generate new ones. The
% mutation operator is applied to the population to sustain diversity,
% by changing the value of a part of a solution. Both crossover and
% mutation operators are applied to explore new points in the search
% space~\cite{blickle1995mathematical}.

The Tournament Selection is a very popular selection operator for
Genetic Algorithms and Evolutionary Computation in general. Some of
the reasons for this popularity are that it is very simple to
implement, and that it has a single control parameter, the
\emph{Tournament Size}, also known as ``Tournament's $k$''.

How should the value of the Tournament Size be chosen? Early works
claimed that the value of $k$ could be used to control the balance
between exploration and exploitation in the
search~\cite{blickle1995mathematical}. As $k$ gets larger, the
selection pressure is expected to increase, and better solutions are
expected to exert a larger influence in the following
generations~\cite{miller1995genetic}.

On the other hand, works in the past 15 years have commonly used very
small values for $k$, such as 2 or
3~\cite{deb2000efficient,beyer2001self,
  kaelo2007integrated,nicolau2009application,
  sawyerr2011comparative,sawyerr2015benchmarking,
  oztekin2018decision}. The reason for this change is hard to
identify, as most papers either do not justify their choice of
parameters, or lightly mention a concern over computational cost for
this rule of thumb.

%One of the most used selection operator is the tournament selection
%operator. It selects solutions based on their fitness value and on one
%parameter, the tournament size. Controlling the tournament size
%parameter is argued to be one way of adjusting the balance between the
%exploration and explotation~\cite{blickle1995mathematical}, which
%governs the search process in GA. Although it is a very important
%balance, it cannot be set directly~\cite{filipovic2012fine}. The
%selection pressure of tournament selection is expected to increase as
%the tournament size becomes larger. That is, the higher the selection
%pressure, the more the better solutions influences the next
%generation~\cite{miller1995genetic}.

%As a rule of thumb, small values, such as 2 or 3, are widely
%used. Filipovic~\cite{filipovic2012fine} argued that small vales lead
%to small selection pressure and may be good choices, but, very often,
%the search process converges too slowly with smaller tournament size
%(and too fast with bigger tournament size).

Precious few works look too closely at their choice of tournament
size. One example is Nicolau~\cite{nicolau2009application} who tried
to empiricaly define a relationship between population size and
tournament size for fine tuning a binary GA.

In order to question and further understand these assumptions and
rules of thumb, we present an experimental study on the role of the
tournament selection size. Our study focuses on a real-valued GA
applied to the noise-free Black Box Optimization Benchmark (BBOB)
functions.

We vary the mutation operator (uniform or SBX), the generational
scheme, the modality of the objective function (unimodal or
multimodal), and the dimensionality of the objective function (10, 20,
40).  We observe how the quality of the final answer changes as the
value of $k$ changes.

Our results show that for functions with only one local optimum
(easier problems), the choice of $k$ does not show an effect on the
final result. For harder problems, when the functions have many local
optimum we saw an effect. The exact best value of $k$ may depend on
the function, dimensionality and algorithm composition and this
relationship is not clear. Instead of using $k$ equals to 2 or 3, we
recommend that the choice of this parameter be considered more
carefully.
