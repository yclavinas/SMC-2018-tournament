\documentclass[sigconf, authordraft, anonymous]{acmart}

\usepackage{booktabs} % For formal tables
\usepackage{subcaption}
\usepackage[]{algorithm2e}

% Copyright
%\setcopyright{none}
%\setcopyright{acmcopyright}
%\setcopyright{acmlicensed}
\setcopyright{rightsretained}
%\setcopyright{usgov}
%\setcopyright{usgovmixed}
%\setcopyright{cagov}
%\setcopyright{cagovmixed}


% DOI
\acmDOI{10.1145/nnnnnnn.nnnnnnn}

% ISBN
\acmISBN{978-x-xxxx-xxxx-x/YY/MM}


%Conference
\acmConference[GECCO '18]{the Genetic and Evolutionary Computation
Conference 2018}{July 15--19, 2018}{Kyoto, Japan}
\acmYear{2018}
\copyrightyear{2018}

%\acmArticle{4}
\acmPrice{15.00}

\begin{document}
\title{Experimental Analysis of the Tournament Size on Genetic Algorithms}

\author{Yuri Lavinas}
\affiliation{\institution{University of Tsukuba}}
\email{yclavinas@gmail.com}

\author{Claus Aranha}
\affiliation{\institution{University of Tsukuba}}
\email{caranha@cs.tsukuba.ac.jp}

\author{Tetsuya Sakurai}
\affiliation{\institution{University of Tsukuba}}
\email{sakurai@cs.tsukuba.ac.jp}

\begin{abstract}
We perform an experimental study about the effect of the tournament
size parameter from the Tournament Selection operator. Tournament
Selection is a classic operator for Genetic Algorithms and Genetic
Programming. It is simple to implement and has only one control
parameter, the \emph{tournament size}. Even though it is commonly
used, most practitioners still rely on rules of thumb when choosing
the tournament size. For example, almost all works in the past 15
years use a value of 2 for the tournament size, with little reasoning
behind that choice. To understand the role of the tournament size, we
run a real-valued GA on 24 BBOB problems with 10, 20 and 40
dimensions. We also vary the crossover operator and the generational
policy of the GA. For each combination of the above factors we observe
how the quality of the final solution changes with the tournament
size. Our findings do not support the indiscriminate use of tournament
size 2, and recommend a more careful set up of this parameter.
\end{abstract}

\begin{CCSXML}
<ccs2012>
<concept>
<concept_id>10010147.10010257.10010293.10011809.10011812</concept_id>
<concept_desc>Computing methodologies~Genetic algorithms</concept_desc>
<concept_significance>500</concept_significance>
</concept>
</ccs2012>
\end{CCSXML}

\ccsdesc[500]{Computing methodologies~Genetic algorithms}

\keywords{Genetic Algorithm, Tournament Selection, Selection Schemes}


\maketitle

%\input{samplebody-conf}
%%%%%%%%%%%%%%%%%%%%%%%%%%%%%%%%%%%%%%%%%%%%%%%%%%%%%%%%%%%%%%%%%%
\section{Introduction}\label{intro}

% Genetic Algorithm (GA) is a well-known meta-heuristic that is used to
% seek solutions given a search space. The GA relies on operators such
% as the crossover operator, the mutation operator and the selection
% operator. These operators have their own and specific role in the
% GA. The selection operator selects solutions to compose the next
% generation given their measured fitness function quality value, with
% higher solutions having higher probability to compose the next
% generation. The crossover operator generates new solutions by
% recombining pairs of good solutions to generate new ones. The
% mutation operator is applied to the population to sustain diversity,
% by changing the value of a part of a solution. Both crossover and
% mutation operators are applied to explore new points in the search
% space~\cite{blickle1995mathematical}.

The Tournament Selection is a very popular selection operator for
Genetic Algorithms and Evolutionary Computation in general. Some of
the reasons for this popularity are that it is very simple to
implement, and that it has a single control parameter, the
\emph{Tournament Size}, also known as ``Tournament's $k$''.

How should the value of the Tournament Size be chosen? Early works
claimed that the value of $k$ could be used to control the balance
between exploration and exploitation in the
search~\cite{blickle1995mathematical}. As $k$ gets larger, the
selection pressure is expected to increase, and better sollutions are
expected to exert a larger influence in the following
generations~\cite{miller1995genetic}.

On the other hand, works in the past 17 years have commonly used very
small values for $k$, such as 2 or
3~\cite{deb2000efficient,beyer2001self,
  kaelo2007integrated,nicolau2009application,
  sawyerr2011comparative,sawyerr2015benchmarking, oztekin2018decision}. The reason for this
change is hard to identify, as most papers either do not justify their
choice of parameters, or lightly mention a concern over computational
cost for this rule of thumb.

%One of the most used selection operator is the tournament selection
%operator. It selects solutions based on their fitness value and on one
%parameter, the tournament size. Controlling the tournament size
%parameter is argued to be one way of adjusting the balance between the
%exploration and explotation~\cite{blickle1995mathematical}, which
%governs the search process in GA. Although it is a very important
%balance, it cannot be set directly~\cite{filipovic2012fine}. The
%selection pressure of tournament selection is expected to increase as
%the tournament size becomes larger. That is, the higher the selection
%pressure, the more the better solutions influences the next
%generation~\cite{miller1995genetic}.

%As a rule of thumb, small values, such as 2 or 3, are widely
%used. Filipovic~\cite{filipovic2012fine} argued that small vales lead
%to small selection pressure and may be good choices, but, very often,
%the search process converges too slowly with smaller tournament size
%(and too fast with bigger tournament size).

Precious few works look too closely at their choice of tournament
size. One example is Nicolau~\cite{nicolau2009application} who tried
to empiricaly define a relationship between population size and
tournament size for fine tuning a binary GA.

In order to question and further understand these assumptions and
rules of thumb, we present an experimental study on the role of the
tournament selection size. Our study focuses on a real-valued GA
applied to the noise-free Black Box Optimization Benchmark (BBOB)
functions.

We vary the mutation operator (uniform or SBX), the generational
scheme, the modality of the objective function (unimodal or
multimodal), and the dimensionality of the objective function
(10, 20, 40).  We observe how the runtime and the quality of the final
answer changes as the value of $k$ changes.

Our results show that... and can be used to understand that...

%TODO: Add here the referes for "from the literature
% A review of the literature shows that there is a preference to small
% values for the tournament size, as 2 or 3 ~\cite{goldberg1991real,
%   goldberg1993toward, agrawal1995simulated, harik1999gambler,
%   tsutsui1999multi, harik1999compact, deb2000efficient,
%   beyer2001self,kaelo2007integrated, bhunia2009application,
%   nicolau2009application, sawyerr2011comparative,
%   sawyerr2015benchmarking}. This has been a popular choice since
% 1991. Most of the papers use little scientific justification for
% their choice for the tournament size value. On the other hand,
% Nicolau~\cite{nicolau2009application} tried to define empirically the
% value of the tournament size for his binary GA.
 
 

%TODO: include the ref to bbob functions
\section{The Tournament Selection}\label{sec:background:Selection Scheme} 


A selection scheme is a process that aims to improve the average
quality of the GA by giving solutions of higher quality a higher
probability of composing the next
generation~\cite{blickle1995mathematical}.  The tournament selection,
as any selection mechanism in GA, simply favors the selection of
better individuals of a population to influence the next generation.

Tournament Selection is a selection scheme that has a unique set of
characteristics that makes it a very popular
choice~\cite{blickle1995mathematical}. It is simple to code, and its
complexity is O(n) on the size of the
population~\cite{goldberg1991comparative} which makes it efficient for
both nonparallel and parallel architectures.

 The tournament selection is also translation and scaling
 invariant~\cite{maza1993analysis}, which means that the behavior of
 the selection scheme is not affected by the scale of the fitness
 values~\cite{back2000evolutionary}.


It is often claimed that the tournament selection is able to adjust
the selection pressure by changing its only parameter, which controls
the number of individuals chosen to take part in the tournament. That
is an important feature, because any selection scheme should have the
capacity to adjust its selection pressure given different
domains. When only two individuals are selected, it is often called a
binary tournament~\cite{blickle1995mathematical,
  goldberg1991comparative}, which is a popular choice. However, in
principle there is no limit for the size of the tournament.

\subsection{The Tournament Selection Procedure}\label{sec:background:tournament_selection} 



Tournament selection is conducted as: choose a number $K$ of solutions randomly from the population and select the best solution from this group as a parent for the crossover operator. Repeat this operation for each parent necessary. This $K$ is the tournament size parameter that we investigate in this paper.




\subsection{Selection Schemes and BBOB functions}



Holtschulte and Moses~\cite{holtschulte2013benchmarking} used rank selection in their GA implementation and applied it to the BBOB functions. They declared that rank selection should be replaced by an elitism selection scheme to improve the performance of their GA implementation by spreading of high quality solutions through the population. 

The roulette-wheel and the tournament selection are two of these elitist selection schemes. However, it has been shown that roulette-wheel leads to premature convergence~\cite{baker1987reducing}. 

Nicolau, ~\cite{nicolau2009application} used the tournament selection in his GA with the goal of solving the BBOB functions. He declares ``for harder problem domains, such as the ones on the BBOB-Benchmark suite, the tournament selection scheme is more appropriate.''.

These two studies in addition to the tournament selection popularity and its power to be adjusted for different domains, give us motive to analyze the tournament selection behavior considering the BBOB functions.


\subsection{Tournament Pressure}\label{sec:background:tournament_pressure} 
Muhlenbein et al., in~\cite{muhlenbein1993predictive}, introduced the concept of selection intensity, which is to measure the pressure of selection schemes. The selection intensity is defined as the expected average fitness of the population after selection. This definition can only be applied because the tournament selection is scale and translation invariant~\cite{blickle1995mathematical}.

Blickle et. al~\cite{blickle1995mathematical}, declare that the selection intensity can be analytically calculated for tournament sizes of 2 and 3 and for larger values with numerical integration. They calculate the selection intensity for values between 2 and 30 and showed that with higher values for the tournament size the selection intensity increases. This calculation allows the prediction of the convergence time and can be used to compare different selection schemes.

Blickle et .al~\cite{blickle1995comparison} compared some selection schemes, given their selection intensity. They concluded that for the same selection intensity the tournament selection has the smallest loss of diversity and the highest selection variance when compared with the truncation selection, the linear ranking, the exponential rank, and the proportional rank.

Miller et al.~\cite{miller1995genetic}, argued that with higher tournament sizes, the selection pressure rises resulting in higher convergence rates, because the convergence rate of a GA is largely determined by the selection pressure.  Accordingly, if the selection pressure is too low, the convergence rate will be slower if it is too high, there is an increased chance of the GA prematurely converging.

Based on the studies from above, we understand that the convergence directly influences the the final results of a GA, once any variation on the convergence rates changes the lost of diversity, and the selection variance. Consequently, it is important to explore the values for tournament size, since this value seems to determine the balance of exploration and exploitation by changing the selection intensity. It is possible that this balance could vary given a domain, once it is expected that every domain holds a specific characteristic that would need a different proportion of this balance.

\subsection{Tournament Size in the Literature}\label{sec:background:tournament_size} 

In 1991, Goldberg~\cite{goldberg1991real}, analyzed the tournament selection with size 2 in a real-coded GA. He recognized empirically that after some time only individuals with relatively high function values would be represented in the population. He recognizes that selection may emphasize highly fit individuals leading to a small number of alternatives considered.

Agrawal~\cite{agrawal1995simulated}, in 1995, implemented real-coded GA with tournament selection with size 2. Their goal was to compare the results of the simulated binary crossover (SBX) with the binary single-point crossover and the blend crossover. They concluded that the SBX has search power similar to the single-point crossover and the SBX performs better in difficult test functions than the blend crossover. They understand that for some functions the selection pressure is not enough.

In 1999, Tsutsui et al.~\cite{tsutsui1999multi} proposed a simplex crossover (SPX), a multi-parent recombination operator for real-coded genetic algorithms. For their experiments, they used a real-coded GA with tournament selection with size 2 and showed that ``SPX works well on functions having multi modality and/or epistasis with a medium number of parents: 3-parent on a low dimensional function or 4 parents on high dimensional functions.''. 

Later, in 2000, Deb~\cite{deb2000efficient}, implemented a binary GA and a real-coded GA, with tournament size 2. He aimed to devise a penalty function approach that does not require any penalty parameter. Because all of their search spaces are defined in the real space, the real-coded GA was more suited in finding feasible solutions, but in all cases, this approach `` [has] a niche over classical methods to handle constraints''. The concluded with declaring that some parametric study should be done on the important parameters to improve the performance.

%Bhunia et al.~\cite{bhunia2009application}, in 2009, created a tournament GA application to solve an economic production lot-size (EPL) model. In their study, the tournament selection is used to produce a better population from the existing two populations. 


In 2001, Beyer et al.~\cite{beyer2001self}, analyzed the results of using the simulated binary crossover (SBX), the fuzzy recombination operator and the blend crossover (BLX), all with no mutation and tournament size 2 in a real-coded GA. The results indicated that those crossovers exhibit similar performance and linear convergence order.


Sawyerr et al.~\cite{sawyerr2015benchmarking}, in 2015, proposed the RCGAu, hybrid real-coded genetic algorithm with ``uniform random direction'' search mechanism. They used tournament selection with size 3 to explore the suite of noiseless black-box optimization testbed and the results showed great performance in some functions such as $f_7$ and $f_{21}$.

Goldberg et AL~\cite{goldberg1993toward}, in 1993, implemented a binary GA using the tournament selection as a convenient way to control selection pressure. They wanted to study the theory aspects of the GA focusing in the mixing building block. The actual effect of the tournament value was not investigated empirically in the work.

In 1999, Harik et al.~\cite{harik1999compact} studied a simple GA and experimented with the tournament size. Their work shows that a given solution quality can be obtained faster by the tournament size equal to 4 or 8, instead of tournament size equal to 2 or 40. 

Also, Harik et al.~\cite{harik1999gambler} implement a binary coded GA using tournament selection. They analyze the impact of the tournament size on their GA with a 100-bit one-max function. They inspected this implementation with tournament sizes of 2, 4, and 8. They observed that the proportion of the building blocks changed with the tournament size, being higher for smaller tournament sizes.

In 2007, Kaelo et al.~\cite{kaelo2007integrated} compared the usage of the tournament size in a real-coded GA. They discuss the impact of using tournament size of 2 and tournament size of 3. The results of comparing these two tournament sizes indicated that the tournament with size 3 is more greedy, reduces the number of functions evaluations greatly and also improves CPU usage, therefore, it impacts negatively in the success rate of their GA.


Also in 2009, Nicolau~\cite{nicolau2009application} implemented a binary GA to solve the noise-free BBOB 2009 testbed. Trough a quick experimentation with a subset of benchmark functions which yielded the following equation:

%\vspace{-7.5mm}
\begin{equation}
k = P/500, \text{ where P is the population size.}
\end{equation}


From the works reviewed here, it is possible to understand that there is a preference to small values for the tournament size. We traced back to 1991, and we found out that it has been a popular choice ever since. We must highlight that these values are usually chosen without much justification.




 
\section{Noise free benchmark BBOB functions}

The noise free benchmark BBOB functions~\cite{hansen2010real} are a selected group of functions that are often used to evaluate the performance of evolutionary computation algorithms.


These functions can be separated in two groups by their complexity. The first group is composed on unimodal functions while the second group is composed on multimodal functions. Here we consider that unimodal functions are functions with only one local optimum (equal to the global optimum) while multimodal functions are functions that have many local optimum (with one of them being the global optimum). 

The unimodal functions used here are: Sphere Function, Ellipsoidal Function (F2), Linear Slope, Atractive Sector Function, Step Ellipsoidal Function, Rosenbrock Function - original, Rosenbrock Function - rotated, Ellipsoidal Function (F10), Discus Function, Bent Cigar Function, Sharp Ridge Function, and Different Powers Function.

While the multimodal functions used here are: Rastrigin Function (F3), Buche-Rastring Function, Rastring Function (F15), Weierstrass Function, Schaffers F7 Function, Schaffers F7 Function moderately ill-conditioned, Composite Griewank-Rosenbrock Function F8F2, Schwefel Function, Gallagher's Gaussian 101-me Peaks Function, Gallagher's Gaussian 21-hi Peaks Function, Katsuura Function, and Lunacek bi-Rastrigin Function.

\section{Experiment Design}

\subsection{The Genetic Algorithm}\label{sec:proposed:ga}

To test the influence of the tournament size we implemented a simple real-valued Genetic Algorithm to explore the search space of the 24 noise free N dimension BBOB benchmark functions~\cite{hansen2010real}, while changing the value of the tournament size.

\subsubsection*{Genome Representation and Evolutionary Tools.}
Each solution is represented as real valued array, where each element is one input to a N dimension noise free benchmark BBOB function. Therefore, each possible solution of the GA has size of N real value elements. The values of any solution are defined to the be into the interval [-5, 5], since the target functions are bounded by this interval. We considerer two crossover operators when experimenting with the GA: the Uniform crossover and the Simulated Binary crossover (SBX). The other evolutionary operators are elitism and Gaussian mutation. The relevant parameters were set as in the Table~\ref{relevant_par} and Table~\ref{gaussian_par}.

\vspace{3mm}
\begin{table}[!ht]
	\centering
	\begin{tabular}{|l|l|}
	\hline
	Number of Evaluations & 40000 \\ \hline
	Population Size &  800		\\ \hline
	Crossover chance 	& 0.9	\\ \hline
	Mutation Chance 	& 0.1	\\ \hline		
	Elitism size 		& 1		\\ \hline		
	\end{tabular}
	\caption{GA Relevant Parameters}
	\label{relevant_par}
\end{table}
	\vspace{-2mm}
%
\begin{table}[!ht]
	\centering
	\begin{tabular}{|l|l|}
	\hline
		Mean & 0 \\ \hline		
		Standard Deviation & 1 \\ \hline		
		Ind. Prob. Attribute to be Mutated &  0.1 \\ \hline		
	\end{tabular}
	\caption{Gaussian Mutation parameters}
	\label{gaussian_par}
\end{table}

\subsubsection*{Fitness Function.}
The fitness function considered are the N dimensions noise free benchmark BBOB functions~\cite{hansen2010real}.





\subsubsection*{Generational Scheme.}
Two generation schemes were used, the first is named the ($\lambda, \lambda$), also called the comma version, and the second is named the ($\lambda + \lambda$), also called the plus version. The ($\lambda, \lambda$) scheme has a $\lambda$ population of parents and for every generation it creates the $\lambda$ offspring to compose the next population. The ($\lambda + \lambda$) scheme creates the $\lambda$ offspring and the best $\lambda$ individuals from combined population of offspring and parents are select to the next generation~\cite{pilat2017parallel}.

Code for reproduction of the experiment is available at [anonymous].

%comentario para claus re-escrever
%TODO
\subsection{Experiments}\label{sec:experiment}
We made two experiments aiming to verify the impact of different values for the tournament size. In the first experiment, we analyzed the relation between the tournament size and the performance on the BBOB benchmark functions with 10, 20 and 40 dimensions, when combined with the Uniform crossover or with the SBX, with the generational scheme  ($\lambda, \lambda$). Then, we analyzed the same relation from before with the same functions and dimensions, but with the generational scheme  ($\lambda + \lambda$) and only the SBX crossover. We consider that the impact the results may be different acknowledged the 2 groups of functions and therefore we analyze the results by grouping all functions together, or the unimodal functions or the multimodal functions. 

We used the Friedman Test to determine whether any of the values for the tournament size show a significant difference in the average performance. In each of these tests, we set $\alpha = 0.05$. For this analysis, the dependent variable being the average performance, the blocking variable is the BBOB function number and the treatment variable is the tournament size value.




\section{Results}\label{sec:results}

\begin{table}[h]
	\centering
	\begin{tabular}{|l|l|l|l|}
		\hline
		\textbf{Function Group} & \textbf{Dimension size}      & \textbf{Chi-squared}        & \textbf{P-value}                     \\ \hline
		\multicolumn{1}{|l|}{Unimodal} & \multicolumn{1}{|l|}{10} & \multicolumn{1}{l|}{24.955} & \multicolumn{1}{l|}{ 0.2992} \\ \hline
		\multicolumn{1}{|l|}{Multimodal} & \multicolumn{1}{|l|}{10} & \multicolumn{1}{l|}{66.904} & \multicolumn{1}{l|}{2.012e-06}  \\ \hline
		\hline
		\multicolumn{1}{|l|}{Unimodal} & \multicolumn{1}{|l|}{20} & \multicolumn{1}{l|}{20.24} & \multicolumn{1}{l|}{0.5681} \\ \hline
		\multicolumn{1}{|l|}{Multimodal} & \multicolumn{1}{|l|}{20} & \multicolumn{1}{l|}{57.525} & \multicolumn{1}{l|}{5.152e-05}  \\ \hline
		\hline	
		\multicolumn{1}{|l|}{Unimodal} & \multicolumn{1}{|l|}{40} & \multicolumn{1}{l|}{25.34} & \multicolumn{1}{l|}{0.2811} \\ \hline
		\multicolumn{1}{|l|}{Multimodal} & \multicolumn{1}{|l|}{40} & \multicolumn{1}{l|}{37.091} & \multicolumn{1}{l|}{0.02312}  \\ \hline
	\end{tabular}
	\caption{Friedman Test results for Uniform Crossover - ($\lambda, \lambda$) scheme.}
	\label{Friedman_test_uniform-a}	
\end{table}


\begin{table}[h]
	\centering
	\begin{tabular}{|l|l|l|l|}
		\hline
		\textbf{Function Group} & \textbf{Dimension size}      & \textbf{Chi-squared}        & \textbf{P-value}                     \\ \hline
		\multicolumn{1}{|l|}{Unimodal} & \multicolumn{1}{|l|}{10} & \multicolumn{1}{l|}{53.379} & \multicolumn{1}{l|}{0.0002011} \\ \hline
		\multicolumn{1}{|l|}{Multimodal} & \multicolumn{1}{|l|}{10} & \multicolumn{1}{l|}{81.653} & \multicolumn{1}{l|}{8.645e-09}  \\ \hline
		\hline
		\multicolumn{1}{|l|}{Unimodal} & \multicolumn{1}{|l|}{20} & \multicolumn{1}{l|}{113.02} & \multicolumn{1}{l|}{3.181e-14} \\ \hline
		\multicolumn{1}{|l|}{Multimodal} & \multicolumn{1}{|l|}{20} & \multicolumn{1}{l|}{88.885} & \multicolumn{1}{l|}{5.302e-10}  \\ \hline
		\hline
		\multicolumn{1}{|l|}{Unimodal} & \multicolumn{1}{|l|}{40} & \multicolumn{1}{l|}{163.13} & \multicolumn{1}{l|}{$>$ 2.2e-16} \\ \hline
		\multicolumn{1}{|l|}{Multimodal} & \multicolumn{1}{|l|}{40} & \multicolumn{1}{l|}{118.46} & \multicolumn{1}{l|}{3.316e-15}  \\ \hline
	\end{tabular}
	\caption{Friedman Test results for SBX Crossover - ($\lambda, \lambda$) scheme.}
	\label{Friedman_test_sbx-a}	
\end{table}

	

\begin{table}[h]
	\centering
	\begin{tabular}{|l|l|l|l|}
		\hline
		\textbf{Function Group} & \textbf{Dimension size}      & \textbf{Chi-squared}        & \textbf{P-value}                     \\ \hline
		\multicolumn{1}{|l|}{Unimodal} & \multicolumn{1}{|l|}{10} & \multicolumn{1}{l|}{54.011} & \multicolumn{1}{l|}{0.0001639} \\ \hline
		\multicolumn{1}{|l|}{Multimodal} & \multicolumn{1}{|l|}{10} & \multicolumn{1}{l|}{75.724} & \multicolumn{1}{l|}{8.073e-08}  \\ \hline
		\hline
		\multicolumn{1}{|l|}{Unimodal} & \multicolumn{1}{|l|}{20} & \multicolumn{1}{l|}{73.078} & \multicolumn{1}{l|}{2.15e-07} \\ \hline
		\multicolumn{1}{|l|}{Multimodal} & \multicolumn{1}{|l|}{20} & \multicolumn{1}{l|}{95.524} & \multicolumn{1}{l|}{3.867e-11}  \\ \hline
		\hline
		\multicolumn{1}{|l|}{Unimodal} & \multicolumn{1}{|l|}{40} & \multicolumn{1}{l|}{115.38} & \multicolumn{1}{l|}{1.198e-14} \\ \hline
		\multicolumn{1}{|l|}{Multimodal} & \multicolumn{1}{|l|}{40} & \multicolumn{1}{l|}{87.318} & \multicolumn{1}{l|}{9.762e-10}  \\ \hline
	\end{tabular}
	\caption{Friedman Test results for SBX Crossover - ($\lambda + \lambda$) scheme.}
	\label{Friedman_test_sbx-b}	
\end{table}



\begin{figure*}[!t]
%	\Large{Average performance on different tournament size - Gallagher's Gaussian 21-hi Peaks Function}
	\begin{subfigure}[b]{0.33\textwidth}
		\centering
		\includegraphics[width=\textwidth]{img/SBX-10D/multimodal_sbx_22_dim_10.pdf}
		\caption{Gallagher's Gaussian 21-hi Peaks Function - 10 dimensions.}
	\end{subfigure}
	\begin{subfigure}[b]{0.33\textwidth}
		\centering
		\includegraphics[width=\textwidth]{img/SBX-20D/multimodal_sbx_22_dim_20.pdf}
		\caption{Gallagher's Gaussian 21-hi Peaks Function - 20 dimensions.}
	\end{subfigure}
	\begin{subfigure}[b]{0.33\textwidth}
		\centering
		\includegraphics[width=\textwidth]{img/SBX-40D/multimodal_sbx_22_dim_40.pdf}
		\caption{Gallagher's Gaussian 21-hi Peaks Function - 40 dimensions.}
	\end{subfigure}
	\caption{SBX crossover - ($\lambda, \lambda$) scheme.}
	\label{sbx-22-a}
	\begin{subfigure}[b]{0.33\textwidth}
		\centering
		\includegraphics[width=\textwidth]{img/uniform-10D/multimodal_uniform_22_dim_10.pdf}
		\caption{Gallagher's Gaussian 21-hi Peaks Function - 10 dimensions.}
	\end{subfigure}
	\begin{subfigure}[b]{0.33\textwidth}
		\centering
		\includegraphics[width=\textwidth]{img/uniform-20D/multimodal_uniform_22_dim_20.pdf}
		\caption{Gallagher's Gaussian 21-hi Peaks Function - 20 dimensions.}
	\end{subfigure}
	\begin{subfigure}[b]{0.33\textwidth}
		\centering
		\includegraphics[width=\textwidth]{img/uniform-40D/multimodal_uniform_22_dim_40.pdf}
		\caption{Gallagher's Gaussian 21-hi Peaks Function - 40 dimensions.}
	\end{subfigure}
	\caption{Uniform crossover - ($\lambda, \lambda$) scheme.}
	\label{uniform-22-a}
	\begin{subfigure}[b]{0.33\textwidth}
		\centering
		\includegraphics[width=\textwidth]{img/2n2n-10D/multimodal_2n2n_22_dim_10.pdf}
		\caption{Gallagher's Gaussian 21-hi Peaks Function - 10 dimensions.}
	\end{subfigure}
	\begin{subfigure}[b]{0.33\textwidth}
		\centering
		\includegraphics[width=\textwidth]{img/2n2n-20D/multimodal_2n2n_22_dim_20.pdf}
		\caption{Gallagher's Gaussian 21-hi Peaks Function - 20 dimensions.}
	\end{subfigure}
	\begin{subfigure}[b]{0.33\textwidth}
		\centering
		\includegraphics[width=\textwidth]{img/2n2n-40D/multimodal_2n2n_22_dim_40.pdf}
		\caption{Gallagher's Gaussian 21-hi Peaks Function - 40 dimensions.}
	\end{subfigure}
	\caption{SBX crossover - ($\lambda + \lambda$) scheme.}
	\label{sbx-22-B}
\end{figure*}


\begin{figure*}[!t]
	\begin{subfigure}[b]{0.33\textwidth}
		\centering
		\includegraphics[width=\textwidth]{img/SBX-10D/unimodal_sbx_11_dim_10.pdf}
		\caption{Discus Function - 10 dimensions.}
	\end{subfigure}
	\begin{subfigure}[b]{0.33\textwidth}
		\centering
		\includegraphics[width=\textwidth]{img/SBX-20D/unimodal_sbx_11_dim_20.pdf}
		\caption{Discus Function - 20 dimensions.}
	\end{subfigure}
	\begin{subfigure}[b]{0.33\textwidth}
		\centering
		\includegraphics[width=\textwidth]{img/SBX-40D/unimodal_sbx_11_dim_40.pdf}
		\caption{Discus Function - 40 dimensions.}
	\end{subfigure}
	\caption{SBX crossover - ($\lambda, \lambda$) scheme.}
	\label{sbx-11-a}
	\begin{subfigure}[b]{0.33\textwidth}
		\centering
		\includegraphics[width=\textwidth]{img/uniform-10D/unimodal_uniform_11_dim_10.pdf}
		\caption{Discus Function - 10 dimensions.}
	\end{subfigure}
	\begin{subfigure}[b]{0.33\textwidth}
		\centering
		\includegraphics[width=\textwidth]{img/uniform-20D/unimodal_uniform_11_dim_20.pdf}
		\caption{Discus Function - 20 dimensions.}
	\end{subfigure}
	\begin{subfigure}[b]{0.33\textwidth}
		\centering
		\includegraphics[width=\textwidth]{img/uniform-40D/unimodal_uniform_11_dim_40.pdf}
		\caption{Discus Function - 40 dimensions.}
	\end{subfigure}
	\caption{Uniform crossover - ($\lambda, \lambda$) scheme.}
	\label{uniform-11-a}
	\begin{subfigure}[b]{0.33\textwidth}
		\centering
		\includegraphics[width=\textwidth]{img/2n2n-10D/unimodal_2n2n_11_dim_10.pdf}
		\caption{Discus Function - 10 dimensions.}
	\end{subfigure}
	\begin{subfigure}[b]{0.33\textwidth}
		\centering
		\includegraphics[width=\textwidth]{img/2n2n-20D/unimodal_2n2n_11_dim_20.pdf}
		\caption{Discus Function - 20 dimensions.}
	\end{subfigure}
	\begin{subfigure}[b]{0.33\textwidth}
		\centering
		\includegraphics[width=\textwidth]{img/2n2n-40D/unimodal_2n2n_11_dim_40.pdf}
		\caption{Discus Function - 40 dimensions.}
	\end{subfigure}
	\caption{SBX crossover - ($\lambda + \lambda$) scheme.}
	\label{sbx-11-B}
\end{figure*}


\begin{figure*}[!t]
%	\captionsetup{format = hang, justification = raggedright}
	\begin{subfigure}[b]{0.24\textwidth}
		\centering
		\includegraphics[width=\textwidth]{img/covergency_multimodal_2n2n_16_dim_20_tsize_16.pdf}
		\caption{Weierstrass Function with the SBX crossover - $(\lambda + \lambda)$ - 20 dimensions}
	\end{subfigure}
	\begin{subfigure}[b]{0.24\textwidth}
		\includegraphics[width=\textwidth]{img/covergency_unimodal_sbx_8_dim_10_tsize_9.pdf}
		\caption{Rosenbrock Function - original with the SBX crossover - $(\lambda, \lambda)$ - 10 dimensions}
	\end{subfigure}
	\begin{subfigure}[b]{0.24\textwidth}
		\centering
		\includegraphics[width=\textwidth]{img/covergency_multimodal_sbx_4_dim_10_tsize_5.pdf}
		\caption{Buche-Rastrigin Function with the SBX crossover - $(\lambda + \lambda)$ - 10 dimensions}
	\end{subfigure}
	\begin{subfigure}[b]{0.24\textwidth}	
		\includegraphics[width=\textwidth]{img/covergency_multimodal_sbx_3_dim_40_tsize_4.pdf}		
		\caption{Rastrigin Function with the SBX crossover - $(\lambda + \lambda)$ - 40 dimensions}
	\end{subfigure}
	\label{convergence}

	\caption{Examples of convergence for different \\combinations of functions and generational schemes.}
		\raggedright
\end{figure*}



\subsection{Overall Effect of K}

The Friedman Test showed no effect of changing the value of K
for unimodal functions with the uniform crossover and any
number of dimensions (Table~\ref{Friedman_test_uniform-a}). 

On the other hand, for unimodal functions with the SBX crossover, and
for multimodal functions with any other setup, the test indicated a
significant effect of changing the value of K on the final result of
the search~(Tables~\ref{Friedman_test_sbx-a}, \ref{Friedman_test_sbx-b}). 

An example of this effect can be seen in Figure~\ref{uniform-22-a},
where it is possible to see that for the Gallagher's Gaussian 21-hi
Peaks Function higher values for the tournament size indicate better
results.

\subsection{Analysis by Function}

To get a finer intuition of these results, we show some visual
examples separated in two groups of figures. The first group shows the
mean value achieved by the GA for the Gallagher's Gaussian 21-hi Peaks
Function, and the second group does the same for the Discus
Function. Both groups show the mean function values of 40 repetitions
for each set of parameters, and a linear regression line on the value
of K against the mean function value. While the results discussed here
are representative for the other functions studied, similar figures
for all other functions are available on the source repository.

Figures~\ref{sbx-22-a} and ~\ref{uniform-22-a} show that for the
Gallagher's Gaussian 21-hi Peaks Function (with 10, 20 and 40
dimensions) changing the tournament size to higher values tends to
increase the quality of the results for the generational scheme
($\lambda, \lambda$). On the other hand, for the generational scheme
($\lambda + \lambda$), Figure~\ref{sbx-22-B} shows that changing the
dimension has a dominating impact on the final values found by the
GA. This means that in this case the best tournament value depends on
the number of dimensions in the problem.

The dominance of the dimensionality observed in Figure~\ref{sbx-22-B}
also occurs in the Discus Function.
Figures~\ref{sbx-11-a},~\ref{uniform-11-a}, (with 10, 20 and 40
dimensions) show that changing the dimensionality of the problem may
have a high impact in the choice of the tournament value with the
generational scheme ($\lambda, \lambda$). Figure~\ref{sbx-11-B} shows
that with the generational scheme ($\lambda + \lambda$) changing the
tournament size leads to better final results. From these figures we
understand that smaller values positively affect the results,
specially with low dimensionality.

For all figures, the gray shaded area represents the 95\% confidence
level interval of a linear regression model. From these regressions,
we can see that sometimes choosing small values for K can be a poor
choice, a good choice, or make no difference at all. For all Figures,
the mean of 40 repetitions is shown as bullets and the red line at the
bottom shows the target value for the function.


\subsection{Convergence Assumption}

Finally, we want to make sure that all GA compositions used are actually
performing the search (even though the final effectiveness varies with
parameter choice). To verify this assumption, we perform a
visual examination of the convergence of the functions studied. 

Figure~\ref{convergence} is a selection of randomly samples of this
examination, namely the Weierstrass Function, the Rosenbrock Function
- original, the Buche-Rastrigin Function, and the Rastrigin Function
(F3), respectively.

From these figures we can see that the studied GAs are indeed
converging towards the optimum target value, represented by the bottom
red line. Similar figures for other GA/function combinations are
available on the source repository.


\section{Conclusion}
\label{sec:conclusion}


We proposed an experimental analysis of the impact of the tournament size on the BBOB benchmark functions. We verified that there are little mathematical or experimental studies that supports a choice of value for the tournament size. Usually, the tournament size is chosen to be 2 or 3, based on ``previous results''.
 
 
We analyzed a group of tournament size values on different selection operators and we found that using the tournament size with values like 2 or 3 may not be good choices, depending on the characteristic of the problem you are exploring. For that, we propose that some sort of self-adaptative procedure should be tried, aiming to minimize the impact of bad choice of values for the tournament size.

We are aware that our work is limited to few tournament size values. Also, we understand that the other parameters may have some influence in the results. We know that there is much more that could be done in this field of study, therefore we propose that more investigations need to be done, in order to truly understand the role of the tournament size value related to the quality of final results. Also, we would like to explore whether our results were mainly related with the particular sets of GA operators studied here.


\bibliographystyle{ACM-Reference-Format}
\bibliography{sample-bibliography} 

\end{document}
