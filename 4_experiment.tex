\section{Experiment Design}

\subsection{The Genetic Algorithm}\label{sec:proposed:ga}

To test the influence of the tournament size we implemented a simple
real-valued Genetic Algorithm to explore the search space of the 24
noise free N dimension BBOB benchmark functions~\cite{hansen2010real},
while changing the value of the tournament size.

\subsubsection*{Genome Representation and Evolutionary Tools.}

Each solution is represented as a N-dimensional real valued array,
which is used as the input parameters to a N dimensional noise free
BBOB function.  The search space for each function is bounded to the
[-5, 5] interval for every dimension.

We considerer two crossover operators when experimenting with the GA:
the Uniform crossover and the Simulated Binary crossover (SBX). The
other evolutionary operators are elitism and Gaussian mutation. The
relevant parameters were set as in the Table~\ref{relevant_par} and
Table~\ref{gaussian_par}.

\vspace{3mm}
\begin{table}[!ht]
	\centering
	\begin{tabular}{|l|l|}
	\hline
	Number of Evaluations & 40000 \\ \hline
	Population Size &  800		\\ \hline
	Crossover chance 	& 0.9	\\ \hline
	Mutation Chance 	& 0.1	\\ \hline		
	Elitism size 		& 1		\\ \hline		
	\end{tabular}
	\caption{GA Relevant Parameters}
	\label{relevant_par}
\end{table}
	\vspace{-2mm}
%
\begin{table}[!ht]
	\centering
	\begin{tabular}{|l|l|}
	\hline
		Mean & 0 \\ \hline		
		Standard Deviation & 1 \\ \hline		
		Ind. Prob. Attribute to be Mutated &  0.1 \\ \hline		
	\end{tabular}
	\caption{Gaussian Mutation parameters}
	\label{gaussian_par}
\end{table}

\subsubsection*{Fitness Function.}
The fitness function considered are the N dimensional noise free
benchmark BBOB functions~\cite{hansen2010real}.

\subsubsection*{Generational Scheme.}
Two generation schemes were used, the first is named the ($\lambda,
\lambda$), also called the comma version, and the second is named the
($\lambda + \lambda$), also called the plus version. The ($\lambda,
\lambda$) scheme has a $\lambda$ population of parents and for every
generation it creates the $\lambda$ offspring to compose the next
population. The ($\lambda + \lambda$) scheme creates the $\lambda$
offspring individuals from the combined population
of offspring and parents and the best $\lambda$ are select to the next
generation. For more information refer to Pil{\'a}t publication~\cite{pilat2017parallel}.

Code for reproduction of the experiment is available at github.com/yclavinas/gaParameterAnalysis.

%comentario para claus re-escrever
%TODO
\subsection{Experiment}\label{sec:experiment}

Our experiment aims to find out whether the value of K makes a
consistent impact on the performance of GA, and if it does, whether
this impact is dependent on the problem being solved, and on the
composition of the GA. To answer this question, we follow this procedure:

First, we perform GA searches while varying the following factors:
Function (the 24 unimodal and multimodal noise-free BBOB), Dimension
Size (10, 20, 40), Crossover Operator (SBX or Uniform), Generational
Scheme ($\lambda, \lambda$) or ($\lambda + \lambda$) and Tournament
Size (2 to 25). For each combination above, we repeat the GA search 40
times, and the mean final function evaluation value is chosen as the
representative value for that particular combination.

We group the experiments by the modality of the function (unimodal or
multimodal), the dimension and the GA composition, and for each group
perform an inference test on the value of the tournament size. On each
group we perform a Friedman Test to determine whether any of the
values for K show a significant difference in the average
performance. In each test, we set $\alpha = 0.05$, the dependent
variable being the mean final function evaluation value, the blocking
variable is the BBOB function and the treatment variable is the
tournament size.




